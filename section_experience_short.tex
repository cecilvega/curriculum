%Section: Work Experience at the top
\sectionTitle{Experiencia laboral}{\faSuitcase}
%\renewcommand{\labelitemi}{$\bullet$}
\begin{experiences}

    \experience
        {Noviembre 2021}
        {Data Scientist}
        {Komatsu}
        {Actualidad}
        {
            \begin{itemize}
            \item Desarrollo indicador de severidad operativa de camiones mineros utilizando datos de falla de componentes a nivel latinoamericano.
            Indicador se obtiene mediante técnicas predictivas que permiten conectar el evento de falla con las variables más importantes que influyen en la operación minera.
            \item Desarrollo y construcción de infraestructuras de datos utilizando Airflow como orquestador de datos.
            Enfoque en el desarrollo y traspaso de conocimiento respecto a la modularización de desarrollos, orquestación de rutinas y adopción de versionamiento de código.
            \item Desarrollo de un sistema de detección automática de anomalías en la condición de los equipos y componentes.
            Esta detección se realiza mediante diversas reglas basadas en conocimiento experto para identificar desviaciones en los datos que se reciben desde las múltiples fuentes, generando alertas y permitiendo la visualización integrada de las distintas variables.
            \end{itemize}
        }
        {
            Python,
            Spark,
            Airflow,
            Azure,
            Pytorch,
            MLFlow
        }

    \experience
        {Enero 2021}
        {Data Scientist - Área Monitoreo}
        {Komatsu}
        {Noviembre 2021}
        {
            \begin{itemize}
            \item Desarrollo de algoritmo de detección por imágenes del estado de los tapones magnéticos acorde a su depósito ferromagnético.
            \item Cálculo de impacto en indicadores de confiabilidad del área.
            \item Automatización de reportabilidad, y generación de visualizadores.
            \item Análisis multivariable para detección de anomalías.
            \end{itemize}
        }
        {
            Python,
            SQL,
            Dagster,
            PowerBI
        }
    \experience
    {Noviembre 2021}
    {Data Scientist - Área Planificación}
    {Komatsu}
    {Enero 2020}
    {
        \begin{itemize}
            \item Key User RMCARE, sistema que concentra la información de de eventos y tareas asociadas al mantenimiento.
            Generación de infraestructura de datos para la visualización de indicadores de planificación y confiabilidad.
            Enfoque en asegurar integridad y calidad de la información a lo largo de su ciclo de vida.
            \item Minería de texto a partir de información obtenida en eventos y tareas a nivel Nacional.
            Información utilizada para apoyar la toma de desiciones en la gestión del pool de componentes.
        \end{itemize}
    }
    {
        Python,
        R,
        SQL,
        Dagster,
        DBT,
        Tableau
    }
        \experience
        {Julio 2019}
        {Data Scientist}
        {Contac Ingenieros}
        {Enero 2020}
        {
            \begin{itemize}
                \item Optimización de rendimiento mina mediante localización de regiones propensas a la generación de fallas. Enfoque en visualización y presentación de análisis de datos.
                \item Apoyo en proyecto sensor virtual para predicción de particulado de la molienda.
            \end{itemize}
        }
        {
            Python,
            R
        }


  

\end{experiences}
