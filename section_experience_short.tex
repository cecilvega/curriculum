%Section: Work Experience at the top
\sectionTitle{Experiencia laboral}{\faSuitcase}
%\renewcommand{\labelitemi}{$\bullet$}
\begin{experiences}

    \experience
        {Noviembre 2021}
        {Data Scientist}
        {Komatsu}
        {Actualidad}
        {
            \begin{itemize}
            \item Desarrollo indicador de severidad operativa de camiones mineros utilizando datos de falla de componentes a nivel latinoamericano. Indicador se obtiene mediante técnicas predictivas que permiten conectar el evento de falla con las variables más importantes que influyen en la operación minera.
            \item Principal contribuyente en la adopción de Airflow como orquestador de datos. Enfoque en el desarrollo y traspaso de conocimiento respecto a la modularización de desarrollos, orquestación de rutinas y adopción de versionamiento de código.
            \item Desarrollo de un sistema de detección automática de anomalías en la condición de los equipos y componentes. Esta detección se realiza mediante diversas reglas basadas en conocimiento experto para identificar desviaciones en los datos que se reciben desde las múltiples fuentes, generando alertas y permitiendo la visualización integrada de las distintas variables.
            \end{itemize}
        }
        {
            Airflow,
            Azure,
            Python, 
            Pytorch
        }

    \experience
        {Enero 2021}
        {Data Engineer}
        % {Ingeniero Soporte Técnico y Diagnóstico Remoto}
        {Komatsu}
        {Noviembre 2021}
        {
            \begin{itemize}
            \item Desarrollo de algoritmo de detección por imágenes del estado de los tapones magnéticos acorde a su depósito ferromagnético.
            \item Generación de visualizadores y automatización de rutinas de datos tal como la emisión de reportabilidad a partir de PDF entregados por Minera Escondida.
            \item Desarrollo de modelos de prognósis para predicción de la vida remanente útil. Enfoque en datos de tribología y telemetría.
            \item Key User RMCARE, sistema que concentra la información de de eventos y tareas asociadas al mantenimiento. Enfoque en la automatización del soporte de la plataforma. 
            \item Extracción de información a partir de grandes volúmenes de datos de texto escrito por los mantenedores. Información utilizada para apoyar la toma de desiciones en la gestión del pool de componentes.
            \end{itemize}
        }
        {
            Python,
            PowerBI
        }
        \experience
        {Julio 2019}
        {Data Scientist}
        {Contac Ingenieros}
        {Enero 2020}
        {
            \begin{itemize}
                \item Optimización de rendimiento mina mediante localización de regiones propensas a la generación de fallas. Enfoque en visualización y presentación de análisis de datos.
                \item Apoyo en proyecto sensor virtual para predicción de particulado de la molienda.
            \end{itemize}
        }
        {
            Python,
            R
        }
    % \experience
    %     {Enero 2020}
    %     {Ingeniero Soporte Planificación}
    %     {Komatsu}
    %     {Enero 2021}
    %     {
    %         \begin{itemize}
            
    %         \end{itemize}
    %     }
    %     {
    %         Python,
    %         SQL
    %         }

            \experience
        {Enero 2023}
        {Práctica Profesional en Machine Learning Operations}
        {Pinwheel}
        {Abril 2023}
        {
            \begin{itemize}
            \item agregar como proyecto externo
            \item Desarrollo de un stack que permite el entrenamiento de modelo de lenguaje natural basado en grandes volúmenes de datos. Implementación de MLFlow para comparar métricas de distintas versiones de modelos.
            \end{itemize}
        }
        {
            Amazon,
            MLFlow,
            DVC
        }
    % \experience
    %     {Agosto 2017}
    %     {Servicios de Ingeniería para proyecto FONDEF ID15I10496}
    %     {Unidad de desarollo tecnológico (UDT)}
    %     {Junio 2018}
    %     {
    %         \begin{itemize}
    %         \item Desarrollo y redacción como autor principal de investigación de sistemas de almacenamiento de calor latente en el área de bioenergía.
    %         \item Tesista (Agosto 2017- Enero 2018): Desarrollo de un sistema de pruebas para materiales de cambio de fase aplicado en almacenamiento térmico de energía a baja temperatura.
    %         \newline 
    %         Profesor: Alfredo Gordon S. Encargado UDT: Cristina Segura C.
    %         \end{itemize}
    %     }
    %     {
    %         OpenFOAM,
    %         C++
    %     }
    % \emptySeparator
    % \experience
    %     {Noviembre 2016}
    %     {Práctica profesional}
    %     {Unidad de desarollo tecnológico (UDT)}
    %     {Marzo 2017}
    %     {
    %         \begin{itemize}
    %             \item Diseño y construcción de un intercambiador de calor a escala de laboratorio.
    %         \end{itemize}
    %     }
    %     {
    %         Autocad,
    %         ANSYS
    %     }
  

\end{experiences}
